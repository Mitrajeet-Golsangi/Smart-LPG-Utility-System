\documentclass[conference]{IEEEtran}
\usepackage{cite}
\usepackage{amsmath,amssymb,amsfonts}
\usepackage{algorithmic}
\usepackage{graphicx}
\usepackage{textcomp}
\usepackage{xcolor}
\usepackage[all]{nowidow}
\usepackage[none]{hyphenat}

\newenvironment{Figure}
  {\par\medskip\noindent\minipage{\linewidth}}
  {\endminipage\par\medskip}

\def\BibTeX{{\rm B\kern-.05em{\sc i\kern-.025em b}\kern-.08em
    T\kern-.1667em\lower.7ex\hbox{E}\kern-.125emX}}
\begin{document}

\title{IoT based LPG Gas Utility system}

\author{
  \IEEEauthorblockN{
    Mitrajeet Golsangi\IEEEauthorrefmark{1},
    Divija Godse\IEEEauthorrefmark{2},
    Vivek Ghuge\IEEEauthorrefmark{3},\\
    Vishwajeet Haralkar\IEEEauthorrefmark{4},
    Adityaraj Honraopatil\IEEEauthorrefmark{5} and
    Dr. Vijay Gaikwad\IEEEauthorrefmark{6}
  }
  \IEEEauthorblockA{
    \textit{dept. of Computer Science} \\
    \textit{Vishwakarma Institute of Technology}\\
    Pune, India\\
    Email : \IEEEauthorrefmark{1}mitrajeet.golsangi20@vit.edu,
    \IEEEauthorrefmark{2}divija.godse20@vit.edu,
    \IEEEauthorrefmark{3}vivek.ghuge20@vit.edu,\\
    \IEEEauthorrefmark{4}vishwajeet.haralkar20@vit.edu,
    \IEEEauthorrefmark{5}adityaraj.honraopatil20@vit.edu,
    \IEEEauthorrefmark{6}vijay.gaikwad@vit.edu,
  }
}


\maketitle

\begin{abstract}
  LPG is a powerful and efficient fuel that is primarily
  used in private cooking areas. LPG is highly combustible
  and can cause a fire even if it is far away from the
  source of the leak. The majority of fires are caused by
  a faulty rubber tube or when the regulator is not turned
  off. The proposed system is built around a microcontroller
  that includes gas sensors, GSM, a display, and a buzzer.
  The gas leak will be detected by the sensor, which will
  then send the information to the microcontroller. Based
  on that microcontroller makes a decision and displays a
  warning message. Another issue is the user not being
  able to tell the level of LP gas in the cylinder causing
  problems at the 11th hour. The proposed solution has
  brought a feasible solution to the same causing the
  users to monitor the Gas level from their mobile phones.
\end{abstract}

\begin{IEEEkeywords}
  LPG, IoT, LPG hazards, LPG monitoring , IoT embedded system
\end{IEEEkeywords}

\section{Introduction}
Our aim is to do a complete evaluation of the LPG
gas systems which have been rooted in nearly 30
crore people in India.\cite{[8]} It is a tedious process to
check the level of gas cylinder at home without any
special equipment and book the new one in anticipation.
We aim to improve the safety, usability and effectiveness
of the LPG gas cylinder.\\

LPG gas safety is a very important aspect of the
complete system. Gas leakage leads to various accidents
resulting in both financial loss as well as human
injuries.\cite{[6]} The risk of firing, explosion, suffocation
all are based on their physical properties such as
flammability, toxicity etc. According to the Ministry
of Petroleum and Natural Gas, Government of India
over 835 deaths have been recorded in 2018-19 in
India alone.\cite{[7]} Inspections by oil companies found that
many LPG consumers are unaware of safety checks of gas
cylinders. There is a need for a system to detect and
also prevent leakage of LPG.

\section{Literature Survey}
In these papers, the authors have mentioned the
use of an MQ-6 gas sensor, for the sole purpose
of leakage detection.\cite{[1]} The MQ-6 sensor senses LPG
leakage in any case and sends alert signals to the
microcontroller when it reaches a dangerous level.
Calculation: power of sensitivity\\[0.3em]
$P_s = V_c^2 \times \frac{R_s}{(R_s+R_L)^2}$\\[0.3em]
Resistance of sensor ($R_s$):\\[0.3em]
$R_s=\frac{V_c}{V\times R_L - 1}\times R_L$ \cite{[3]}\cite{[4]}\\
From the microcontroller, the alert signals are then
sent to the users via GSM.\cite{[2]} Thus the users would be
made aware of any kind of LPG leakage. Weight sensor
is used for the purpose of monitoring and detecting
the amount of gas present in the LPG cylinder.\cite{[3]}
Usually, the permissible net weight for LPG in the
domestic cylinder is 14kg. The weight of an empty
cylinder is approximately 15.3kg. The total gross
weight of the cylinder can be rounded off to
29.5kg. Thus when this figure changes the weight
sensor senses the smallest changes which are
further displayed on the LCD. When it reaches
0.5kg the system sends an alert to the user
device.\cite{[4]}

To address the above mentioned issues our project
uses the MQ-135 Gas sensor and alerts the user if the
regular gas composition of the surrounding atmosphere
changes. Our project would also use a 4.08mm force
sensor in combination with the Gas Sensor for
calculating the weight of the cylinder and notifying
the user if a certain threshold is crossed.\cite{[11]}
We intend to combine these two sensors and create a
system that would prevent gas leakage and also serve
as an alert sending device to prevent last minute
shortage.

Many solutions have been in the industry for years like
a solution created by students of KIIT \cite{[7]}.
The proposed solution aims to improve the safety,
usability and effectiveness of the LPG gas cylinder
usage. The advantages of the system are:
As the Weight sensor output is converted to digital
in the reference system there is a loss of accurate
data on actually how much level of gas is remaining
in the cylinder. According to the given paper \cite{[5]}
Arduino UNO has been employed for a similar system.
However working with the Arduino UNO development board has
a certain set of limitations. The proposed system avoids these
problems by using the Node MCU which has a built in
ESP8266 WiFi module the cost of the project has been
reduced By giving the user an application to monitor
their gas systems the accessibility and control over
the system have been increased significantly

\pagebreak

\section{Design of IoT System}
The project implementation is done on an IoT system. Thus, a
suitable design of the same was needed. The following figure
shows the domain model specification of the entire system.
\begin{Figure}
  \centering
  \includegraphics[width=\linewidth]{./Images/DomainModelPPT.drawio.png}
  \figurename{Domain Specification}
\end{Figure}

\section{Methodology}
The proposed system detects the gas above the gas tube of the cylinder.
It also provides users with the ability to monitor gas levels in
LPG cylinders.The following image explains the device component integration
needed to achieve the same.
\begin{Figure}
  \centering
  \includegraphics[width=\linewidth]{Images/BlockDiagram.png}
  \label{block}
  \figurename{Block Diagram}
\end{Figure}
The oversimplified block diagram (fig. \ref{block}) allows the users to
see the basic workflow of the project. The 4051
multiplexer toggles the analog input on the Node MCU
(pin A0) which is controlled by three pins D1, D2, D3
on Node MCU. The maximum amount of time is given for
the gas leakage detection whereas a finite amount of
time (not sure exactly how much yet) is given to the
Measurement of the level of gas in the cylinder. All
the statistics is sent to a host on the local network
and then is fetched from the user's application to give
him/her the most accurate data in real time.
\begin{Figure}
  \centering
  \includegraphics[width=\linewidth]{Images/Simulation.png}
  \label{sim}
  \figurename{Simulation}
\end{Figure}
figure \ref{sim} shows the simulation of the project that
has been performed on Autodesk TinkerCAD®. Here 2 LCD
monitors are used as wireless transmission cannot be
simulated. As seen the readings of the force and gas
Sensor are shown on the user device (LCD Display in
this case) and as the gas has crossed the soft
threshold of 470 PPM. The Yellow LED has been lit
up to warn the user of some abnormal Gas concentrations
in the surrounding atmosphere.

\subsection{Hardware}
\subsubsection{Part Required}
\begin{itemize}
  \item Node MCU 0.9 (ESP8266 Module)
  \item MQ-135 Gas Sensor
  \item A-52A 5.08mm Force Sensor
  \item 4051N Multiplexer IC
  \item Resistors
  \item LED
  \item 15 pin female socket
\end{itemize}
\subsubsection{PCB Creation}
After the gathering of required components the team created a
schematic for the required circuit.
\begin{Figure}
  \centering
  \includegraphics[width=\linewidth]{Images/Schematic.png}
  \label{scheme}
  \figurename{Schematics}
\end{Figure}
After completion of the schematic and checking the connections the PCB board
layout was designed and finalized as shown below

\subsection{Software}
\subsubsection{Used Softwares}
\begin{itemize}
  \item VS Code IDE
  \item Platform IO IDE
  \item Eagle
\end{itemize}

\section{Application Creation}
\subsection{Introduction}
The project uses a mobile application as the means of connecting
the device data and the end user. All the data collected from
the sensors is uploaded to the local network in the form of a
JavaScript Object Notation (JSON) format. This is useful as the
mobile application then collects the data and directly processes
it into useful information and displays it to the users. This exact
process is explained in detail in the paper below

\subsection{Structure}
The application is segregated into 2 main parts
\begin{itemize}
  \item Sensor readings
  \item Network Configuration
\end{itemize}

\textbf{Sensor Readings : }This section displays the collected
sensor data to the user. The sensor data is displayed in the
format of percentage and a graph which is refreshed at 200 ms
for getting the new data.

\textbf{Network Configuration : } This is to ensure that the
ESP8266 WiFi module is connected to the application and correctly
configured.

\subsection{State Management}
In any application where the data needs to be shared from one
screen to another a concept of state management is used.\cite{[10]}
This is needed in order to share data between the various
screens or pages of the application. Flutter has several
state management systems including redux framework and BLoC
architecture\cite{[9]}. In this application BLoC architecture is
used.

\section{BLoC Architecture ?}
BLoC abbreviates to Business Logic of Components. This method is
recommended in the official Flutter documentation by the Flutter's
Development Team. The BLoC state management system is similar to
the M-V-VM (Model View ViewModel) state management system\cite{[12]}, but uses
various streams in order to achieve the same effect. The concept
of BLoC is to have a corresponding effect for each state inside the
application, this is better explained in the figure below.

\begin{Figure}
  \centering
  \includegraphics[width=\linewidth]{Images/BLoC_Example.drawio.png}
  \label{blocFig}
  \figurename{BLoC Workflow}
\end{Figure}

Like displayed in the above figure the UI Screen loads an event on the
event stream which is then carried on to the BLoC. According to this
event package the BLoC interacts with the repositories and emits a
stream of states on different stages of the interaction. This state
is then again carried out to the UI Screen and required changes are
made to indicate to the user some processes are going on in the
backend. This methodology has several advantages.

\subsection{Advantages of BLoC State Management}
\begin{enumerate}
  \item State of application can be known in any point of time
  \item Testing of the application is easily possible
  \item Data driven decisions can be made by recording the user behaviour
  \item Components can be reused making code efficient
\end{enumerate}
The application uses two cubits in order to manage its state.
The main difference between a cubit and a bloc is that the cubits
don't use event streams but send the states on a stream. Meaning
only the response of the cubit is a stream and not the input.

\section{Application Workflow}
The applications workflow is explained in accordance to the figure:

\begin{Figure}
  \centering
  \includegraphics[width=\linewidth]{Images/AppWorkflow1.drawio.png}
  \label{App Workflow}
  \figurename{BLoC Workflow}
\end{Figure}


The first cubit is to get the sensor readings from the API endpoint
generated on the network IP generated by the ESP8266 WiFi module.
This cubit emits 3 states. The initial state sets the force sensor
and the gas sensor readings to zero and continues to send requests
to the API endpoint. The next states are determined by the response
emitted by the repositories. If the response is an error response
code i.e. anything other than the 200 HTTP response code the cubit
emits the Error state and notifies the user some error has occurred,
else the force sensor and gas sensor values are updated giving the user
the correct values of the same. The new requests are made in an interval
of 200 milliseconds, thus the user gets the data in realtime. The
repositories emit two kinds of data, percentage and actual values of
the sensors, the percentage is displayed directly to the user and the
actual values are used to map out a graph of the sensor readings, this gives
a better understanding of the situation to the user.

A similar approach can be observed in the network configuration cubit where
the cubit emits three states the same as the former. These emitted states
are then caught by the UI components and then those components are rerendered
to give the user the necessary information.

\section{Conclusion}
The most common problem in many households of sudden shortage of LPG
gas which is the most essential component of the kitchen has been there
for decades, but no one has tried to fix it. This simple system can have
a great effect not only in the households but also on the larger scale,
like hotels, restaurants and even mobile kitchens. Even though the system's
main focus is to
\bibliographystyle{IEEEtran}
\bibliography{references.bib}

\end{document}
